\documentclass{article} \usepackage[utf8]{inputenc} \title{PS10 
Mozaffar} \author{Yaseen Mozaffar } \date{April 2020} \usepackage{array} 
\begin{document} \maketitle \begin{center} \section{5 Tribes Table} 
\renewcommand{\arraystretch}{1.5} \begin{tabular}{ 
|m{6em}|m{4.2em}|m{3.2em}|m{6em}|m{3em}|m{3em}|m{6em}| }
\hline
 & \textbf{Tree} & \textbf{Log Regress} & \textbf{Neural Net} & 
\textbf{Naive Bayes} & \textbf{KNN} & \textbf{SVM} \\
 \hline
 \textbf{Optimal Parameters} & Split: 47 \newline Bucket: 9 \newline 
cp:.0102 & lambda: 0.181 \newline alpha: 0.027 & size: 8 \newline decay: 
0.362 & N/A & k: 30 & cost: 2 Gamma: 0.25 \\
 \hline
 \textbf{Out of Sample F1} & 0.895 & 0.576 & 0.606 & 0.624 & 0.648 & 
0.903\\
 \hline
 \textbf{Out of Sample G Mean} & 0.665 & 0.665 & 0.696 & 0.734 & 0.741 & 
0.742\\
 \hline \end{tabular} \section{Model Comparison} For both F1 and 
G-Means, a higher score represents a more accurate model. Therefore, we 
can see that, if evaluating off of F1, the models rank as follows: \\ 1: 
SVM \\ 2: Tree Model \\ 3: K Nearest Neighbors \\ 4: Naive Bayes \\ 5: 
Neural Network \\ 6: Logistic Regression\\ \vspace{5mm} %5mm vertical 
space \\ Using G means, the models rank this way:\\ 1: SVM \\ 2: KNN \\ 
3: Naive Bayes \\ 4: Neural Network \\ 5: Logistic Regression\\ 5: Tree 
Model \end{center}
\end{document}
