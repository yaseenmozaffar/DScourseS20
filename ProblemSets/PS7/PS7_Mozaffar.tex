\documentclass{article} \usepackage[utf8]{inputenc} \title{PS7 Mozaffar} 
\author{Yaseen Mozaffar } \date{March 2020} \usepackage{natbib} 
\usepackage{graphicx} \begin{document} \maketitle \section{Problem 6} 
\begin{table}[!htbp]
  \centering
  \caption{}
  \label{} \begin{tabular}{@{\extracolsep{5pt}}lccccccc} 
\\[-1.8ex]\hline \hline \\[-1.8ex] Statistic & \multicolumn{1}{c}{N} & 
\multicolumn{1}{c}{Mean} & \multicolumn{1}{c}{St. Dev.} & 
\multicolumn{1}{c}{Min} & \multicolumn{1}{c}{Pctl(25)} & 
\multicolumn{1}{c}{Pctl(75)} & \multicolumn{1}{c}{Max} \\ \hline 
\\[-1.8ex] logwage & 1,669 & 1.625 & 0.386 & 0.005 & 1.362 & 1.936 & 
2.261 \\ hgc & 2,229 & 13.101 & 2.524 & 0 & 12 & 15 & 18 \\ tenure & 
2,229 & 5.971 & 5.507 & 0.000 & 1.583 & 9.333 & 25.917 \\ age & 2,229 & 
39.152 & 3.062 & 34 & 36 & 42 & 46 \\ \hline \\[-1.8ex] \end{tabular} 
\end{table} Income data tends to be MNAR. This may be due to 
self-reporting bias (people of lower income are less likely to respond 
to surveys asking about income) or some other structural difficulty in 
getting in touch with those of lower income. However, neither of these 
explanations necessarily make sense with this dataset because the 
logwage data is missing in about 25.12 percent of the observations while 
a much lower percentage of observations were missing other variables. If 
it was harder to reach people of a certain income, there likely wouldn't 
be such complete data for the other variables. \bigskip \bigskip 
\bigskip \bigskip \bigskip \bigskip \bigskip \bigskip \bigskip \bigskip 
\bigskip \section{Problem 7} \begin{table}[!htbp] \centering
  \caption{}
  \label{} \begin{tabular}{@{\extracolsep{5pt}}lccc} \\[-1.8ex]\hline 
\hline \\[-1.8ex]
 & \multicolumn{3}{c}{\textit{Dependent variable:}} \\ \cline{2-4} 
\\[-1.8ex] & \multicolumn{3}{c}{logwage} \\ \\[-1.8ex] & (1) & (2) & 
(3)\\ \hline \\[-1.8ex]
 hgc & 0.062$^{***}$ & 0.049$^{***}$ & 0.049$^{***}$ \\
  & (0.005) & (0.004) & (0.004) \\
  & & & \\
 collegenot college grad & 0.146$^{***}$ & 0.160$^{***}$ & 0.160$^{***}$ 
\\
  & (0.035) & (0.026) & (0.026) \\
  & & & \\
 tenure & 0.023$^{***}$ & 0.015$^{***}$ & 0.015$^{***}$ \\
  & (0.002) & (0.001) & (0.001) \\
  & & & \\
 age & $-$0.001 & $-$0.001 & $-$0.001 \\
  & (0.003) & (0.002) & (0.002) \\
  & & & \\
 marriedsingle & $-$0.024 & $-$0.029$^{**}$ & $-$0.029$^{**}$ \\
  & (0.018) & (0.014) & (0.014) \\
  & & & \\
 Constant & 0.639$^{***}$ & 0.833$^{***}$ & 0.833$^{***}$ \\
  & (0.146) & (0.115) & (0.115) \\
  & & & \\ \hline \\[-1.8ex] Observations & 1,669 & 2,229 & 2,229 \\ 
R$^{2}$ & 0.195 & 0.132 & 0.132 \\ Adjusted R$^{2}$ & 0.192 & 0.130 & 
0.130 \\ Residual Std. Error & 0.346 (df = 1663) & 0.311 (df = 2223) & 
0.311 (df = 2223) \\ F Statistic & 80.508$^{***}$ (df = 5; 1663) & 
67.496$^{***}$ (df = 5; 2223) & 67.496$^{***}$ (df = 5; 2223) \\ \hline 
\hline \\[-1.8ex] \textit{Note:} & \multicolumn{3}{r}{$^{*}$p$<$0.1; 
$^{**}$p$<$0.05; $^{***}$p$<$0.01} \\ \end{tabular} \end{table} Across 
the four imputation methods, I got beta1 values of .062, .049, .049, and 
.5, respectively. As the true value was .093, it's more likely than not 
that I made a mistake in methods 2-4. But it's too late to troubleshoot 
my code, so oh well. Also, I don't know why table 2 is doing what it's 
doing. (i.e. being off center and not being under the right heading) My 
best guess is that it's too wide, but I can't figure out how to resize. 
Sorry again but I'm out of time. \section{Problem 8} Progress is slow 
but steady. As my project is my grad research, I'm trying to put in 
extra work upfront on the theoretical basis for the paper before I get 
into the actual data analysis. So, not a whole lot being done using 
tools from this class, but I'm more or less done getting through the lit 
base and identifying my datasets. Ultimately, I think I'll be using 
nonlinear modeling techniques in my project. Because the outcome 
variable is binary (monopoly formation), it may be as simple as a logit 
or probit model. In any case, I'm going to attempt an unsupervised 
modeling portion, in hopes that my model can be used to predict what 
industries/companies are likely to form monopolies in the near future. 
\end{document}
