\documentclass[12pt,english]{article} \usepackage{mathptmx} 
\usepackage{color} \usepackage[dvipsnames]{xcolor} 
\definecolor{darkblue}{RGB}{0.,0.,139.} \usepackage[top=1in, bottom=1in, 
left=1in, right=1in]{geometry} \usepackage[authoryear]{natbib} 
\usepackage{url} \usepackage{booktabs} \usepackage{graphicx} 
\usepackage{pdflscape} \usepackage[unicode=true,pdfusetitle,
 bookmarks=true,bookmarksnumbered=false,bookmarksopen=false,
 breaklinks=true,pdfborder={0 0 0},backref=false,
 colorlinks,citecolor=black,filecolor=black,
 linkcolor=black,urlcolor=black]
 {hyperref} \usepackage[all]{hypcap} % Links point to top of image, 
builds on hyperref \usepackage{breakurl} % Allows urls to wrap, 
including hyperref \linespread{2} \begin{document} \begin{singlespace} 
\title{Price of a Life: An Exploration of Pharmaceutical Pricing 
Policies\thanks{This paper appears as a part of my Independent Graduate 
Research project}} \end{singlespace} \author{Yaseen 
Mozaffar\thanks{Department of Economics, University of Oklahoma.\ 
E-mail~address:~\href{mailto:y.mozaffar@ou.edu}{y.mozaffar@ou.edu}}} % 
\date{\today} \date{April 12, 2020} \maketitle \begin{abstract} 
\begin{singlespace} A short summary of what question the project 
answers, what methods are used, and any policy (or business) 
implications from the findings. \end{singlespace} \end{abstract} 
\vfill{} \pagebreak{} \section{Introduction}\label{sec:intro} Part of 
the challenge inherent in Economics is that not every subject of 
interest can be precisely thought of in terms of money and utility. 
Chief among such subjects is the economic value of human life. This 
paper seeks not to assign a value to life but to demonstrate through a 
case study that, in certain situations, data and economic theory can be 
used to estimate how a particular policy reflects the value placed on 
life by the policymaker. The case study in question is the pricing 
choices made by the manufacturers of life-saving, patent-protected 
drugs. This eliminates many confounding elements that would muddle the 
conclusions that could be properly drawn from the analysis. By drawing 
from monopolies, I eliminate competitive market pressure as a pricing 
factor and, by limiting the analysis to life-saving drugs, I am able to 
identify the relationship between pricing and human life as directly as 
possible. Recognizing that the supply side of a market includes much 
more than the firms, this paper does not assign any share of 
responsibility to any particular actor. [as an example, corporations 
have a responsibility to their shareholders that they remain profitable, 
so that'll influence their decisions. however, that means that their 
shareholders have made a decision to maintain profitability of their 
investments, potentially at the cost of human life. Because there's no 
feasible way to identify all of those factors, I'm limiting the 
conclusions drawn to "somewhere in the supply side of the market" rather 
than specifying the firm or the shareholders or anyone else. Should I 
explain that here or put it somewhere else?] Following the introduction, 
this paper proceeds with a review of the literature surrounding the 
historic value of human life in economic terms, as well as 
pharmaceutical pricing practices. The next section will detail the data 
used to construct this analysis, followed by an explanation of the 
empirical methods, the analysis of the results of those methods, and the 
larger policy recommendations that come about as a result. 
\section{Literature Review}\label{sec:litreview} [I want to hold out on 
my lit review for a bit until my grad research project takes a bit of a 
more definite direction. Much of the lit review will likely be drawn 
from the analysis done on sources used in the other project] 
\section{Data}\label{sec:data} FDA Patent Expiry data will be used to 
identify which drugs will be included in my analysis, as well as a 
preliminary analysis to determine what kind of effect patent expiry and, 
by extension, patents themselves, have on pricing policy. The Federal 
Medicaid pricing database will, of course, be used to track the price of 
drugs over time. Iterations of this database exist over many years, so 
it can be used across the timeframe of the analysis. I'll also look at 
financial report summaries for the drug manufacturers to ensure that, by 
annual profits and dividends paid out, that there exists the capacity 
for these companies to price lower if they so choose. Any company that 
doesn't meet that requirement would be removed from the analysis, 
because I'm trying to isolate decision-makers with as much free rein as 
possible in their pricing decisions. CDC death and mortality data will 
be used to draw the degree to which a particular drug can lower 
mortality for different disease. This will be combined with CDC data on 
the number of people afflicted with a condition in the US to calculate a 
value for how much life would theoretically be saved if a particular 
drug was to be made available to anyone. (more details in the methods 
section) \section{Empirical Methods}\label{sec:methods} In the interest 
of making the most conservative estimate possible, I am going to 
evaluate only the effect of drug pricing on the 27 million uninsured 
americans. (because the argument can be made that companies can price 
higher because insurance companies can pay for it). So what we're 
looking for is the amount of uninsured people afflicted with a 
particular condition who would be likely unable to afford that drug at 
its current price point. Using that data and the differential between 
the status quo and a situation in the case where the drugs are priced at 
production cost, I can estimate how much money has been saved at the 
cost of how many lives There will be an equation to make this clearer. 
[There will be a discussion here on monopolies and how the monopoly 
model informs this model] Model currentPrice: Price of a drug Afflicted: 
how many uninsured people have a particular disease (if exact number is 
unavailable, can be estimated) volumeSold: units of a drug sold in a 
year productionCost: cost to produce a drug canAfford: Estimate of the 
amount of people able to afford that drug in the status quo drugPremium: 
Basically this is a value for how much a particular drug lowers the 
chance of death from a condition 
livesSaved:(Afflicted-canAfford)*drugPremium is a probabilistic value of 
how many lives would theoretically be saved if the drug was available 
freely profits: currentPrice*volumeSold-productionCost 
profits/livesSaved reflects how many dollars the company has gained per 
person killed by the decision to price above production cost. This will 
be calculated at the company level \section{Research 
Findings}\label{sec:results} Yeah this isn't done yet 
\section{Conclusion}\label{sec:conclusion} Basically the thesis of this 
argument isn't really about healthcare or pharmaceuticals or the evils 
of capitalism or something like that. What I try to show here is that 
policymakers/firms/individuals don't get to pretend that "economics" 
absolves responsibility of moral value. If I'm able to trawl through 
datasets and put together a simple model to estimate the tradeoff 
between lives and profits implicitly made by these pricing decisions, 
then firms have the capacity to do the same, with much more precision. 
Therefore, making decisions in terms of only dollar amounts is a choice, 
whether implicitly or explicitly. There are a number of limitations on 
the particular conclusions drawn here (e.g. that executive salaries and 
dividends aren't necessarily elective decisions, because they need to 
attract talent and funding). However, the overall argument remains 
unaffected; whether or not policymakers and firms think in these 
particular terms, the tradeoff is inevitably being made. The only policy 
recommendation is that values are interrogated and reflected in 
decision-making. If pharmaceutical executives was to be confronted with 
the value placed on life by their price decisions and business continued 
as usual, that would be their right. The only difference would be that 
they will have lost the opportunity to pretend they haven't thought 
about the impact that they have. It is my firm belief that if lives were 
explicitly valued, and decisions at every level were to be made in such 
terms, then the corporate structure of our world would look differently. 
But maybe that's wishful thinking. \vfill \pagebreak{} 
\begin{spacing}{1.0} \bibliographystyle{jpe} 
\bibliography{References.bib} \addcontentsline{toc}{section}{References} 
\end{spacing} \end{document}\documentclass[12pt,english]{article} 
\usepackage{mathptmx} \usepackage{color} \usepackage[dvipsnames]{xcolor} 
\definecolor{darkblue}{RGB}{0.,0.,139.} \usepackage[top=1in, bottom=1in, 
left=1in, right=1in]{geometry} \usepackage[authoryear]{natbib} 
\usepackage{url} \usepackage{booktabs} \usepackage{graphicx} 
\usepackage{pdflscape} \usepackage[unicode=true,pdfusetitle,
 bookmarks=true,bookmarksnumbered=false,bookmarksopen=false,
 breaklinks=true,pdfborder={0 0 0},backref=false,
 colorlinks,citecolor=black,filecolor=black,
 linkcolor=black,urlcolor=black]
 {hyperref} \usepackage[all]{hypcap} % Links point to top of image, 
builds on hyperref \usepackage{breakurl} % Allows urls to wrap, 
including hyperref \linespread{2} \begin{document} \begin{singlespace} 
\title{Price of a Life: An Exploration of Pharmaceutical Pricing 
Policies\thanks{This paper appears as a part of my Independent Graduate 
Research project}} \end{singlespace} \author{Yaseen 
Mozaffar\thanks{Department of Economics, University of Oklahoma.\ 
E-mail~address:~\href{mailto:y.mozaffar@ou.edu}{y.mozaffar@ou.edu}}} % 
\date{\today} \date{April 12, 2020} \maketitle \begin{abstract} 
\begin{singlespace} A short summary of what question the project 
answers, what methods are used, and any policy (or business) 
implications from the findings. \end{singlespace} \end{abstract} 
\vfill{} \pagebreak{} \section{Introduction}\label{sec:intro} Part of 
the challenge inherent in Economics is that not every subject of 
interest can be precisely thought of in terms of money and utility. 
Chief among such subjects is the economic value of human life. This 
paper seeks not to assign a value to life but to demonstrate through a 
case study that, in certain situations, data and economic theory can be 
used to estimate how a particular policy reflects the value placed on 
life by the policymaker. The case study in question is the pricing 
choices made by the manufacturers of life-saving, patent-protected 
drugs. This eliminates many confounding elements that would muddle the 
conclusions that could be properly drawn from the analysis. By drawing 
from monopolies, I eliminate competitive market pressure as a pricing 
factor and, by limiting the analysis to life-saving drugs, I am able to 
identify the relationship between pricing and human life as directly as 
possible. Recognizing that the supply side of a market includes much 
more than the firms, this paper does not assign any share of 
responsibility to any particular actor. [as an example, corporations 
have a responsibility to their shareholders that they remain profitable, 
so that'll influence their decisions. however, that means that their 
shareholders have made a decision to maintain profitability of their 
investments, potentially at the cost of human life. Because there's no 
feasible way to identify all of those factors, I'm limiting the 
conclusions drawn to "somewhere in the supply side of the market" rather 
than specifying the firm or the shareholders or anyone else. Should I 
explain that here or put it somewhere else?] Following the introduction, 
this paper proceeds with a review of the literature surrounding the 
historic value of human life in economic terms, as well as 
pharmaceutical pricing practices. The next section will detail the data 
used to construct this analysis, followed by an explanation of the 
empirical methods, the analysis of the results of those methods, and the 
larger policy recommendations that come about as a result. 
\section{Literature Review}\label{sec:litreview} [I want to hold out on 
my lit review for a bit until my grad research project takes a bit of a 
more definite direction. Much of the lit review will likely be drawn 
from the analysis done on sources used in the other project] 
\section{Data}\label{sec:data} FDA Patent Expiry data will be used to 
identify which drugs will be included in my analysis, as well as a 
preliminary analysis to determine what kind of effect patent expiry and, 
by extension, patents themselves, have on pricing policy. The Federal 
Medicaid pricing database will, of course, be used to track the price of 
drugs over time. Iterations of this database exist over many years, so 
it can be used across the timeframe of the analysis. I'll also look at 
financial report summaries for the drug manufacturers to ensure that, by 
annual profits and dividends paid out, that there exists the capacity 
for these companies to price lower if they so choose. Any company that 
doesn't meet that requirement would be removed from the analysis, 
because I'm trying to isolate decision-makers with as much free rein as 
possible in their pricing decisions. CDC death and mortality data will 
be used to draw the degree to which a particular drug can lower 
mortality for different disease. This will be combined with CDC data on 
the number of people afflicted with a condition in the US to calculate a 
value for how much life would theoretically be saved if a particular 
drug was to be made available to anyone. (more details in the methods 
section) \section{Empirical Methods}\label{sec:methods} In the interest 
of making the most conservative estimate possible, I am going to 
evaluate only the effect of drug pricing on the 27 million uninsured 
americans. (because the argument can be made that companies can price 
higher because insurance companies can pay for it). So what we're 
looking for is the amount of uninsured people afflicted with a 
particular condition who would be likely unable to afford that drug at 
its current price point. Using that data and the differential between 
the status quo and a situation in the case where the drugs are priced at 
production cost, I can estimate how much money has been saved at the 
cost of how many lives There will be an equation to make this clearer. 
[There will be a discussion here on monopolies and how the monopoly 
model informs this model] Model currentPrice: Price of a drug Afflicted: 
how many uninsured people have a particular disease (if exact number is 
unavailable, can be estimated) volumeSold: units of a drug sold in a 
year productionCost: cost to produce a drug canAfford: Estimate of the 
amount of people able to afford that drug in the status quo drugPremium: 
Basically this is a value for how much a particular drug lowers the 
chance of death from a condition 
livesSaved:(Afflicted-canAfford)*drugPremium is a probabilistic value of 
how many lives would theoretically be saved if the drug was available 
freely profits: currentPrice*volumeSold-productionCost 
profits/livesSaved reflects how many dollars the company has gained per 
person killed by the decision to price above production cost. This will 
be calculated at the company level \section{Research 
Findings}\label{sec:results} Yeah this isn't done yet 
\section{Conclusion}\label{sec:conclusion} Basically the thesis of this 
argument isn't really about healthcare or pharmaceuticals or the evils 
of capitalism or something like that. What I try to show here is that 
policymakers/firms/individuals don't get to pretend that "economics" 
absolves responsibility of moral value. If I'm able to trawl through 
datasets and put together a simple model to estimate the tradeoff 
between lives and profits implicitly made by these pricing decisions, 
then firms have the capacity to do the same, with much more precision. 
Therefore, making decisions in terms of only dollar amounts is a choice, 
whether implicitly or explicitly. There are a number of limitations on 
the particular conclusions drawn here (e.g. that executive salaries and 
dividends aren't necessarily elective decisions, because they need to 
attract talent and funding). However, the overall argument remains 
unaffected; whether or not policymakers and firms think in these 
particular terms, the tradeoff is inevitably being made. The only policy 
recommendation is that values are interrogated and reflected in 
decision-making. If pharmaceutical executives was to be confronted with 
the value placed on life by their price decisions and business continued 
as usual, that would be their right. The only difference would be that 
they will have lost the opportunity to pretend they haven't thought 
about the impact that they have. It is my firm belief that if lives were 
explicitly valued, and decisions at every level were to be made in such 
terms, then the corporate structure of our world would look differently. 
But maybe that's wishful thinking. \vfill \pagebreak{} 
\begin{spacing}{1.0} \bibliographystyle{jpe} 
\bibliography{References.bib} \addcontentsline{toc}{section}{References} 
\end{spacing}
\end{document}
